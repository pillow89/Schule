\documentclass[12pt, a4paper]{article}
\usepackage[utf8]{inputenc}

\begin{document} 
	 
	\begin{flushleft}

		In dieser Aufgabe muss die Strecke \(\overline{PR}\) gefunden werden.
		\\Gegeben ist: \(\overline{AB}\) = 4.0cm und \(\beta\) = \(72^\circ\).
		\\M ist der Mittelpunkt von \(\overline{AC}\) also \(\overline{AM}\) = \(\overline{CM}\) = die  
		 Hälfte von \(\overline{AC}\)
		\\Außerdem wissen wir, dass wir es hier mit einem gleichschenkligem Dreieck zu tun haben.
		\\Das heißt \(\overline{AC}\) = \(\overline{BC}\).
		\\Und \(\alpha\) = \(\beta\)
			

	\end{flushleft}
	
	\section*{Start}
	
	\begin{flushleft}	
		Zuerst berechnen wir alle Maße dieses Dreiecks. Das heißt alle Seiten und die Winkel.

		\begin{flushleft}
			\textbf{Winkel}
			\\ \(\gamma\) = \(180^\circ -\alpha - \beta\)
			\\ \(\gamma\) = \(180^\circ - 72^\circ - 72^\circ\)
			\\ \(\gamma\) = \(36^\circ\)
		\end{flushleft}		
		
		\begin{flushleft}
			\textbf{Seiten}		
			\\Zuerst \(\overline{AC}\)
			\\ \(\overline{AT}\) = 2,00cm
			\\cos \(\alpha = \frac{\overline{AT}}{\overline{H}}\)
			\\cos \(72^\circ = \frac{2,00cm}{\overline{AC}}\)
			\\Umformen
			\\cos \(72^\circ = \frac{2,00cm}{\overline{AC}}\) \(\vert * \overline{AC} : cos 72^\circ \)
			\smallskip
			\\cos \(72^\circ = \frac{2,00cm}{\overline{AC}}\)
			\smallskip
			\\ \(\overline{AC} = \frac{2,00cm}{cos 72^\circ}\)
			\smallskip
			\\ \(\overline{AC} = 6,47cm = \overline{BC}\)  --- gleichschenkliges Dreieck
			\smallskip
			\\ \(\overline{AM} = 3,235cm = \overline{CM} \)
			\\ Nun berechnen wir die Höhe des Dreiecks. \(\overline{TC}\)
			\\\(\tan \alpha = \frac{\overline{TC}}{\overline{AT}}\)
			\\\(\tan 72^\circ = \frac{\overline{TC}}{\overline{2,00cm}}\)
					\(\vert * 2,00cm\)
			\\\(\overline{TC} = 6,15cm\)			
		\end{flushleft}		
		\begin{flushleft}
			\textbf{PR berechnen}
			\\Berechnen wir als erstes \(\overline{CP}\), da diese Seite einfacher zu berechnen 			  ist.
			\\Da MCP ein rechtwinkliges Dreieck ist, benutzen wir wiedermal den cosinus
			\\cos \(0.5 \gamma = \frac{\overline{MP}}{\overline{CP}}\) 
			\\cos \(18^\circ\) = \(\frac{3,235cm}{\overline{CP}}\) \(\vert * \overline{CP}\) : 					cos 18\(^\circ\)
			\\\(\overline{CP}\) = 3,4cm
			\\ Nun müssen wir \(\overline{TR}\) berechnen. Hier muss man aufpassen.
				Da uns ein Winkel im Dreieck ART fehlt, müssen wir den vorher noch berechnen.
				Wir müssen zuerst mit dem Dreieck AMF arbeiten.
			\\ Zuerst brauchen wir \(\overline{MF}\)
			\\ \(tan \gamma\) = \(\frac{\overline{MC}}{\overline{MF}}\)
			\\ \(tan 36^\circ\) = \(\frac{3,235cm}{\overline{MF}}\)
			\(\vert * \overline{MF}\) : tan 36\(^\circ \)
			\\ \(\overline{MF}\) = 4,45cm
			\\ Uns sind folgende Seiten bekannt: \(\overline{MF}\) und \(\overline{AM}\)		
			\\ Den Winkel \(\alpha 1 \) (Also der obere Teil)
			\\ tan \(\alpha 1\) = \(\frac{\overline{MC}}{\overline{MF}}\)
			\\ \(\alpha 1\) = 36^\(\circ\)
			\\ Wenn man genauer hinsieht, hätte man es durch die vorige Rechnung sehen können, 					dass dies der gleiche Winkel ist, aber sicher ist sicher.
			\\ Berechnen wir nun: \(\overline{TR}\)
			\\ tan \(36^\circ\) = \(\frac{\overline{TR}}{\overline{AT}}\)
			\\ tan \(36^\circ\) = \(\frac{\overline{TR}}{\overline{AT}}\)
				\(\vert * \overline{AT}\)
			\\ \(\overline{TR}\) = 1,45cm
			\\ letzte Rechnung 
			\\ \(\overline{PR}\) = \(\overline{TC}\) - \(\overline{CP}\) - \(\overline{TR}\)
			\\ \(\overline{PR}\) = 6,15cm - 3,4cm - 1,45cm
			\\ \(\overline{PR}\) = \textbf{1,30cm}
		\end{flushleft}
		
	\end{flushleft}	

\end{document}